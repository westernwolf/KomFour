\section{Eigenschaften der Komplexen Zahlen}
Da $\R$ zu $\C$ erweitert wird werden die Grundgesetzte nicht verletzt. Also ist auch $\C$:
\begin{itemize}
\item Kommutativ (Addition \& Multiplikation)
\item Assoziativ (Addition \& Multiplikation)
\item Distributiv
\end{itemize} 

$\C$ erweitert den Zahlenraum nur um die Zahl $j$ (ausserhalb der Elektrotechnik ist $i$ gebräuchlicher), die quadriert $-1$ ergibt.

\section{Dartstellung}
\begin{itemize}
\item Zahlenpaar:
\[
	z=(z_1;z_2) \text{ mit } z_1,z_2\in\R
\]
\item Eulerform:
\[
	z = a + bj\text{ mit } a,b\in\R
\]
\end{itemize}

Eine Zahl $z== a + bj\text{ mit } a,b\in\R$ setzt sich aus folgendem zusammen:
\begin{itemize}
\item[$z$] komplexe Zahl (vom lat. zusammengesetzt)
\item[$j$/$i$] imaginäre Einheit
\item[$a$] Realteil ($=Re(z)$), reelle Zahl
\item[$j\cdot b$] imaginäre Zahl
\item[$b$] Imaginärteil ($=Im(z)$)
\end{itemize}

\section{Grundoperationen}
Zur Vermeidung von Flüchtigkeitsfehlern und zur besseren Lesbarkeit, sollte man
zuerst den Realteil und erst danach den Imaginärteil berechnen, damit das Resultat schon \kom{Aufgeteilt} dasteht. Zudem sollte die imaginäre Einheit $j$ vor den Variablen stehen.
\begin{align*}
&\text{Addition:}\\
&(a_1+ja_2) + (b_1+b_2)= a_1+b_1+ja_2+jb_2 = (a_1+b_1)+j(a_2+b_2) \\
&\text{Subtraktion:}\\
&(a_1+ja_2) - (b_1+b_2)=  a_1-b_1+ja_2-jb_2 = (a_1-b_1)+j(a_2-b_2)   \\
&\text{Multiplikation:}\\
&(a_1+ja_2) \cdot (b_1+b_2)=  a_1\cdot b_1 + \underbrace{j\cdot j}_{-1}\cdot a_2\cdot b_2 
	+ j\cdot a_1\cdot b_2 + j\cdot a_2\cdot b_1 \\
	&\qquad=(a_1\cdot b_1 -  a_2\cdot b_2 )
	+ j(\cdot a_1\cdot b_2 + \cdot a_2\cdot b_1)   \\
&\text{Division:}\\
& (a_1+ja_2) : (b_1+b_2)=\dots=\frac{a_1b_1+a_2b_2}{b_1^2+b_2^2}
	+j\frac{a_2b_1-a_1b_2}{b_1^2+b_2^2}
\end{align*}
Zur Entfernung von $j$ aus dem Nenner erweiter man mit dem Komplexkonjugierten
von $z$ ($z=a+jb \rightarrow \overline{z}=a-jb \text{ mit } a,b\in\R$). 
Ein weiterer Trick ist $\frac{1}{j}=-j$.

% Seite 7 im Skript